\documentclass[11pt,nocut]{article}

\usepackage{../latex_style/packages}
\usepackage{../latex_style/notations}

\title{\vspace{-2.0cm}%
	Optimization and Computational Linear Algebra for Data Science\\
Homework 1: Vector spaces}
\author{\vspace{1cm}Stephen Roy \ $\cdot$ \ \texttt{stephen.i.roy@nyu.edu}}
\date{\vspace{-1cm}Due on September 10, 2019}
\setcounter{section}{1}

\begin{document}
\maketitle
%\noindent\textbf{Rules:}
\centerline{\pgfornament[width=13cm]{89}}
{\small
	\begin{itemize}
		\item Author: \author{Stephen Roy \ $\cdot$ \ \texttt{stephen.i.roy@nyu.edu}}
	\end{itemize}
}
\vspace{-0.4cm}
\centerline{\pgfornament[width=13cm]{89}}
\vspace{0.5cm}



\begin{problem}[2 points]
	Let $u,v$ be two vectors of $\R^2$. Show that either they are linearly dependent or that they span the whole of $\R^2$.
\end{problem}
\begin{proof}
If $u,v$ are linearly dependent we can assume that there exists some scalar $\lambda \in \R^2$ such that:
$$
u = \lambda v
$$
Now that we know $u$ can be written in terms of $v$ we can show that:
$$
\Span(u,v) = \Span(\lambda v,v) = \Span(v) \neq\R^2
$$
By contrast, if $u,v$ are linearly independent (and thus neither can be the \{0\}) we know that:
$$
\Span(u) \neq \Span(v)
$$
Thus the intersection of \Span(u,v) exists in two dimensions, specifically $\R^2$.
\end{proof}	

\vspace{1mm}

\begin{problem}[3 points]
	Are the following sets subspaces of $\R^3$? Justify your answer.
	\begin{enumerate}[label=\normalfont(\textbf{\alph*})]
		\item $E_1= \big\{ (x,y,z) \in \R^3 \, \big| \, x - 2y + z= 0 \big\}$.\\
		\begin{proof} Yes, the line defined by $E_1$ passes through the origin at $E_1=(0,0,0)$ and, as a linear span where $(x,y,z)\in \R^3$ it is a subspace of $\R^3$. More specifically, because all variable terms in $E_1$ are of the first power, we see that vector addition and scalar multiplication hold true over its linear span.\end{proof}
		\item $E_2= \big\{ (x,y,z) \in \R^3 \, \big| \, x - 2y + z= 3 \big\}$.\\
		\begin{proof}No, $E_2=(0,0,0)$ does not include the origin, thus it is not a subspace of $\R^3$.\end{proof}
		\item $E_3= \big\{ (x,y,z) \in \R^3 \, \big| \, 5x + y^2 + z= 0 \big\}$.\\
		\begin{proof}No, the curve $E_3$ can not be extended for any negative $\lambda$ value of $y$ (which when squared become positive), thus it is not a subspace of $\R^3$.\end{proof}
	\end{enumerate}
\end{problem}

\vspace{1mm}

\begin{problem}[3 points]\label{prob:add_vector}
	Suppose that $v_1, \dots, v_k \in \R^n$ are linearly independent. Let $x \in \R^n$ and assume that $x \not\in \Span(v_1,\dots, v_k)$.
	Show that $(v_1,\dots,v_k,x)$ are linearly independent.
\end{problem}
\begin{proof}
Given that $x \not\in \Span(v_1, \dots, v_k)$ we know, by the definition of span, that there exists no $\lambda \in \R^n$ such that any $\lambda v_i$ coincides with $x$. We can demonstrate this via contradiction by assuming there is a $\lambda_{k+1}x \neq 0$ such that the linear combination of $\lambda \{v_1, \dots, v_k, x\}=0$ \\
$$
\lambda_1 v_1 + \dots + \lambda_k v_k + \lambda_{k+1} x = 0
$$
$$
\frac{\lambda_1}{\lambda_{k+1}} v_1 + \dots + \frac{\lambda_k}{\lambda_{k+1}} v_k+ x = 0
$$\\
This means that:\\
$$
-\frac{\lambda_1}{\lambda_{k+1}} v_1 - \dots - \frac{\lambda_k}{\lambda_{k+1}} v_k= x
$$
But this means we have now written x as a linear combination of $(v_1,\dots,v_k)$ which contradicts our given than $x \not\in \Span(v_1, \dots, v_k)$. Thus, because we cannot establish a $\lambda$ which cancels out $x$ using only $\lambda (v_1, \dots, v_k)$ and we know $(v_1, \dots, v_k)$ is already linearly independent, we can establish that $(v_1, \dots, v_k, x)$ must also be linearly independent and the only solution to
$$
\lambda_1 v_1 + \dots + \lambda_k v_k + \lambda_{k+1} x = 0
$$
is 
$$
\lambda_1 = \dots = \lambda_k = \lambda_{k+1} = 0
$$
This confirms that the set $\{v_1,\dots,v_k,x\}$ is linearly independent.
\end{proof}
\vspace{1mm}

\begin{problem}[2 points]
	Let $S$ be a subspace of $\R^n$ and $v_1, \dots, v_k \in S$.
	We assume that $v_1, \dots, v_k$ are linearly independent.
	Show (using the result of Problem~\ref{prob:add_vector}) that one can find vectors $v_{k+1}, \dots, v_{k+m}$ in $S$ such that $(v_1, \dots, v_{k+m})$ is a basis of $S$.
\end{problem}
\begin{proof}
To prove set $S$ is a basis we must demonstrate that it is linearly independent and that its Span is equal to the entire vector space (in this case $\R^n$). In the event that $S$ is already the basis for $\R^n$, then there are no additional vectors that can be added to the set while maintaining linear independency. Otherwise, as seen by the result of Problem~\ref{prob:add_vector} we can add additional vectors and still maintain a linearly independent set of elements within $\R^n$. To find the basis you continue to add $m$ new vectors until you arrive at at set where the $\Span(v_1,\dots,v_k,\dots, v_{k+m}) = \R^n$. Once this occurs you have obtained a basis for $S$.
\end{proof}

\vspace{1mm}

\begin{problem}[$\star$]
	Let $U$ and $V$ be two subspaces of $\R^n$. Show that if
	$$
	\dim(U) + \dim(V) > n,
	$$
	then there must exist a non-zero vector in their intersection, i.e.\ $U \cap V \neq \{0\}$.
\end{problem}
\begin{proof}
Given that $U$ and $V$ are subspaces of $\R^n$, we know that their intersection must also be a subspace (by definition) between size V + W (smallest) and $\R^n$ (largest). Furthermore given $\dim(U) + \dim(V) > n$, we can use the definition of intersection to establish that at least one non-zero vector exists:
$$
\dim(U + V) = \dim(U) + \dim(V) - \dim(U \cap V)
$$
By using the maximum dimension for $\R^n$ we see that:
$$
\dim(\R^n) = n \leq \dim(U + V) = \dim(U) + \dim(V) - \dim(U \cap V)
$$
$$
n \leq \dim(U) + \dim(V) - \dim(U \cap V)
$$
Subtracting n from both sides ($\dim(U) + \dim(V) > n$) shows:
$$
0 \leq c - \dim(U \cap V)
$$
$$
\dim(U \cap V) \leq c
$$
Thus we've established that some constant $c$ must exist which is greater than or equal to the dimension of $U \cap V$. To satisfy this condition, for any non-zero subspace, at least one non-zero vector must exist.
\end{proof}
\vspace{1cm}
\centerline{\pgfornament[width=7cm]{87}}
\end{document}
