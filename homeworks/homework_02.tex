\documentclass[11pt,nocut]{article}

\usepackage{../latex_style/packages}
\usepackage{../latex_style/notations}

\title{\vspace{-2.0cm}%
	Optimization and Computational Linear Algebra for Data Science\\
Homework 2: Linear transformations \& matrices}
\date{\vspace{-1cm}Due on September 17, 2019}
\setcounter{section}{2}

\begin{document}
\maketitle
%\noindent\textbf{Rules:}
\centerline{\pgfornament[width=13cm]{89}}
{\small
	\begin{itemize}
		\item Author: \author{Stephen Roy \ $\cdot$ \ \texttt{stephen.i.roy@nyu.edu}}
	\end{itemize}
}
\vspace{-0.4cm}
\centerline{\pgfornament[width=13cm]{89}}
\vspace{0.5cm}


\begin{problem}[2 points]
	Which of the following are linear transformations? Justify.
	\begin{enumerate}[label=\normalfont(\textbf{\alph*})]
		\item 
			$\displaystyle
			T: \left| 
			\begin{array}{ccc}
				\R^2 & \to & \R^2 \\
				(x,y) & \mapsto & (x^2+y^2, x-y)
			\end{array}
		\right.
		$\\
		
\begin{proof}
		%question a
		No, given $(x_1,y_1) = (0,-2) \hspace{1mm} and \hspace{1mm} (x_2,y_2) = (-2,1)$ we see via contradiction that \\
		$$
		T = 
		\begin{pmatrix}
		x_1+x_2 \\
		y_1+y_2
		\end{pmatrix}
		\neq
		T
		\begin{pmatrix}
		x_1 \\
		y_1
		\end{pmatrix}
		+ T
		\begin{pmatrix}
		x_2 \\
		y_2
		\end{pmatrix}
		$$

		$$
		T = 
		\begin{pmatrix}
		-2 \\
		-1
		\end{pmatrix}
		\neq
		T
		\begin{pmatrix}
		0 \\
		-2
		\end{pmatrix}
		+ T
		\begin{pmatrix}
		-2 \\
		1
		\end{pmatrix}
		$$
				$$
		\begin{pmatrix}
		5 \\
		-1
		\end{pmatrix}
		\neq
		\begin{pmatrix}
		2 \\
		2
		\end{pmatrix}
		+ 
		\begin{pmatrix}
		5 \\
		-1
		\end{pmatrix}
		=
		\begin{pmatrix}
		7 \\
		-1
		\end{pmatrix}
		$$
		
\end{proof}
		
		\item 
			$\displaystyle
			T: \left| 
			\begin{array}{ccc}
				\R^2 & \to & \R^2 \\
				(x,y) & \mapsto & (x+y+1, x-y)
			\end{array}
		\right.
		$\\
		
\begin{proof}
		%question (b)
		$$
		No, \hspace{1mm} T(0,0) \to (1,0). 
		$$
		Furthermore, for T(x,y) to pass through the origin there must be some (x,y) such that (x+y+1) = 0 and (x-y) = 0. Solving the linear equations we find no such (x,y) exist, thus T(x,y) never passes through the origin.
\end{proof}
		
	\item 
			$\displaystyle
			T: \left| 
			\begin{array}{ccc}
				\R^{n \times m} & \to & \R^{m \times n} \\
				A & \mapsto & A^{\sT}
			\end{array}
		\right.
		\quad$ where $A^{\sT}$ is transpose of $A$, i.e. the $m \times n$ matrix defined by
		$$
		\big(A^{\sT}\big)_{i,j} = A_{j,i} \qquad \text{for all} \quad (i,j) \in \{1, \dots, m\}\times \{1, \dots, n\}.
		$$
		
\begin{proof}
		%question (c)
		Yes. Given A, B $\in \R^{m \times n}$:
		$$
		((A+B)^T)_{i,j} = A_{j,i}+B_{j,i} = A^T + B^T
		$$
		$$
		(\lambda A^T)_{i,j} = \lambda A_{j,i} = \lambda (A^T)_{i,j}
		$$
		As both the additive and scalar multiplication properties for the transpose of A (from A) hold true, it is a valid linear transformation.
\end{proof}
		
	\item 
			$\displaystyle
			T: \left| 
			\begin{array}{ccc}
				\R^{n \times n} & \to & \R \\
				A & \mapsto & \Tr(A)
			\end{array}
		\right.\quad$ where $\Tr(A)$ is the trace of the matrix $A$, defined by 
		$$\Tr(A) = \sum_{i=1}^n A_{i,i} \,.$$
		
\begin{proof}
		%question (d)
		Yes. Given A, B $\in \R^{n \times n}$:
		$$
		Tr(A+B) = \sum_{i=1}^n (A_{i,i}+B_{i,i}) = \sum_{i=1}^n A_{i,i} + \sum_{i=1}^n B_{i,i} = Tr(A) + Tr(B)
		$$
		$$
		Tr(\lambda A) = \sum_{i=1}^n \lambda A_{i,i} = \lambda \sum_{i=1}^n A_{i,i}  = \lambda Tr(A)
		$$
		As both the additive and scalar multiplication properties for the Trace of A (from A) hold true, it is a valid linear transformation.
\end{proof}
		
	\end{enumerate}
\end{problem}

\vspace{5mm}

\begin{problem}[3 points]
	Let $f: \R^2 \to \R^3$ be a linear transformation such that
	$$
	f(1,2) = (1,2,3)
	\qquad \text{and} \qquad f(2,2) = (1,0,1).
	$$
	\begin{enumerate}[label=\normalfont(\textbf{\alph*})]
		\item Compute the matrix (canonically) associated to $f$.\\
		\begin{proof}
		%question (a)
		f is a 3 x 2 matrix whose columns are described by:
		$$ f = 
			\begin{pmatrix}
			\vdots & \vdots \\
			f_1 & f_2 \\
			\vdots & \vdots
			\end{pmatrix}
		        \in \R^{3x2}
		$$
		$$
		f(1,2) = (1,2,3) \Rightarrow f_1 + 2f_2 = (1,2,3)
		$$
		$$
		f(2,2) = (1,0,1) \Rightarrow 2 f_1 + 2 f_2 = (1,0,1)
		$$
		Now we have two equations with two variables, solving the system of equations gives:
		$$
		f_1 = (0,-2,-2)
		$$
		$$
		f_2 = ( \frac{1}{2} , 2, \frac{5}{2} )
		$$
		$$ f = 
			\begin{pmatrix}
			0 & \frac{1}{2}\\
			-2 & 2\\
			-2 & \frac{5}{2}
			\end{pmatrix}
		$$
		\end{proof}
		\item Compute the set $\{x \in \R^2 \, | \, f(x) = (1,4,5) \}$.\\
		\begin{proof}
		%question (b)
		$$ f = 
			\begin{pmatrix}
			0 f_1& \frac{1}{2}f_2\\
			-2 f_1& 2f_2\\
			-2 f_1& \frac{5}{2}f_2
			\end{pmatrix}
		=
			\begin{pmatrix}
			1\\
			4\\
			5
			\end{pmatrix}
		$$
		Solving the system of equations gives $f_1 = 0$ and $f_2 = 2$ thus x = (0,2)
		\end{proof}
		\item Compute the set $\{x \in \R^2 \, | \, f(x) = (2,4,5) \}$.\\
		\begin{proof}
		%question (c)
		$$ f = 
			\begin{pmatrix}
			0 f_1& \frac{1}{2}f_2\\
			-2 f_1& 2f_2\\
			-2 f_1& \frac{5}{2}f_2
			\end{pmatrix}
		=
			\begin{pmatrix}
			2\\
			4\\
			5
			\end{pmatrix}
		$$
		The system of equations does not resolve, suppose $f_2 = 4$, thus resolving row 1. Then $f_1$ for row 2 = 2 and row 3 = $\frac{5}{2}$. As 2 $\neq \frac{5}{2}$ there is no solution x such that f(x) = (2,4,5).
		\end{proof}
	\end{enumerate}
\end{problem}
\vspace{5mm}

\begin{problem}[2 points]
	Let $B \in \R^{4 \times 3}$ be a matrix with arbitrary entries:
	$$
	B = 
	\begin{pmatrix}
		B_{1,1} & B_{1,2} & B_{1,3}\\
		B_{2,1} & B_{2,2} & B_{2,3}\\
		B_{3,1} & B_{3,2} & B_{3,3}\\
		B_{4,1} & B_{4,2} & B_{4,3}
	\end{pmatrix}.
	$$
	Find two matrices $A$ and $C$ such that
	$$
	A B C = 
	\begin{pmatrix}
		B_{1,2} & B_{1,1} & B_{1,3} & B_{1,2} \\
		B_{2,2} + B_{3,2} & B_{2,1} + B_{3,1} & B_{2,3} + B_{3,3} & B_{2,2} + B_{3,2} \\
		B_{4,2} & B_{4,1} & B_{4,3} & B_{4,2}
	\end{pmatrix}
	$$
	holds for any $B$ defined above.
\end{problem}
\begin{proof}
	%question 2.3 
	To complete the transformation described above the following actions must take place:
	\begin{enumerate}[label=\normalfont(\textbf{\alph*})]
	\item column 2 goes to column 1
	\item column 1 goes to column 2
	\item column 3 goes to column 3
	\item column 2 goes to column 4
	\item row 3 is added to row 2
	\item row 3 is removed
	\end{enumerate}
	We can complete steps a-d by rearranging columns of the identity matrix as follows:
	$$
		\begin{pmatrix}
		0 & 1 & 0 & 0 \\
		1 & 0 & 0 & 1 \\
		0 & 0 & 1 & 0 \\
		0 & 0 & 0 & 0
		\end{pmatrix}
	$$
	To complete step e, add row 3 to row 2, we add row 3 into row 2 for our modified identify matrix, matrix A:
	$$ A = 
		\begin{pmatrix}
		0 & 1 & 0 & 0 \\
		1 & 0 & 1 & 1 \\
		0 & 0 & 1 & 0 \\
		0 & 0 & 0 & 0
		\end{pmatrix}
	$$
	Then, to complete step f, to transform the dimensions of the matrix and remove row 3,  we can use the identity matrix without the row 3, as matrix, "C":
	$$ C = 
		\begin{pmatrix}
		1 & 0 & 0 & 0 \\
		0 & 1 & 0 & 0 \\
		0 & 0 & 0 & 1
		\end{pmatrix}
	$$
\end{proof}

\begin{problem}[3 points]\leavevmode
	\begin{enumerate}[label=\normalfont(\textbf{\alph*})]
		\item Let $A$ be a $n \times m$ matrix. Show that the image $\Im(A)$ and the kernel $\Ker(A)$ of $A$ are subspaces of respectively $\R^n$ and $\R^m$.
		\item Let
			$$
			A = 
			\begin{pmatrix}
				1 & 2 & 1 & 2 \\
				-1 & 1 & -1 & 1 \\
				0 & 1 & 0 & 2 
			\end{pmatrix}.
			$$
		\begin{proof}
		$$\Ker(A) = {x \in \R^m | Ax = 0}$$
		\begin{enumerate}
			\item Let $u, v \in \Ker(A) \Rightarrow Au = 0$ and $Av = 0$\\
			$$ Au + Av = 0 \Rightarrow A(u+v) = 0 \Rightarrow u + v \in \Ker(A)$$
			\item Let $u \in \Ker(A) \Rightarrow Au = 0$, for any scalar $\alpha \in \R$,\\
			$$ \alpha A u = \alpha . 0 = 0 $$
			$$ A(\alpha u) = 0 \Rightarrow \alpha u \in \Ker(A)$$
			\item A.0 = 0 $\Rightarrow \in \Ker(A)$
		\end{enumerate}
		$$\Im(A) = {Ax \in \R^n | x \in \R^m}$$
		\begin{enumerate}
			\item Let $u, v \in \Im(A) \Rightarrow$, there exists $x, y \in \R^m$ such that Ax = u and Ay = v.
			$$ \Rightarrow Ax + Ay = u + v \Rightarrow A(x+y) = u + v \Rightarrow u + v \in \Im(A)$$
			\item Let u $\in \Im(A) \Rightarrow$, there exists $x \in \R^m$ such that Ax = u, for any scalar $\alpha \in \R$,
			$$ \alpha A x = \alpha . u$$
			$$ \Rightarrow A(\alpha x) = \alpha u \Rightarrow \alpha u \in \Im(A)$$
			\item A.0 = 0 $\Rightarrow 0 \in \Im(A)$
		\end{enumerate}
		\end{proof}
		\vspace{5mm}\\
		Compute a basis of $\Ker(A)$ and show that $\Im(A) = \R^3$.\\
		\vspace{2mm}\\
		\begin{proof}
		$$A = 
			\begin{pmatrix}
				1 & 2 & 1 & 2 \\
				-1 & 1 & -1 & 1 \\
				0 & 1 & 0 & 2 
			\end{pmatrix}
		$$
		Solving the system of linear equations
		$$ 	c1 \begin{pmatrix}
				1 \\
				-1  \\
				0 
			\end{pmatrix}
			+ c2 \begin{pmatrix}
				2 \\
				1  \\
				1  
			\end{pmatrix}
			+ c3 \begin{pmatrix}
				 1  \\
				 -1 \\
				 0  
			\end{pmatrix}
			+ c4 \begin{pmatrix}
				 2 \\
				 1 \\
				 2 
			\end{pmatrix} = 0
		$$
		Yields the following basis for $\Ker(A)$:
		$$ \begin{pmatrix}
				-1 \\
				0 \\
				1 \\
				0 
			\end{pmatrix}
		$$
		Additionally, because we see that columns 1 and 3 are identical, we know know that the $\Im(A)$ or unique column space, can be reduced from $\R^4$ to $\R^3$.
		\end{proof}
	\end{enumerate}
\end{problem}

\vspace{5mm}

\begin{problem}[$\star$]
	Let $A \in \R^{m \times n}$ and $B \in \R^{k \times n}$. 
	Prove that there exists a matrix $C \in \R^{m \times k}$ such that $A = CB$ if and only if $\Ker(B)$ is a subspace of $\Ker(A)$.
\end{problem}
%\begin{proof}
%Answer
%\end{proof}
\vspace{1mm}

\vspace{1cm}
\centerline{\pgfornament[width=7cm]{87}}

%\bibliographystyle{plain}
%\bibliography{./references.bib}
\end{document}